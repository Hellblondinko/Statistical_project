\PassOptionsToPackage{unicode=true}{hyperref} % options for packages loaded elsewhere
\PassOptionsToPackage{hyphens}{url}
%
\documentclass[]{article}
\usepackage{lmodern}
\usepackage{amssymb,amsmath}
\usepackage{ifxetex,ifluatex}
\usepackage{fixltx2e} % provides \textsubscript
\ifnum 0\ifxetex 1\fi\ifluatex 1\fi=0 % if pdftex
  \usepackage[T1]{fontenc}
  \usepackage[utf8]{inputenc}
  \usepackage{textcomp} % provides euro and other symbols
\else % if luatex or xelatex
  \usepackage{unicode-math}
  \defaultfontfeatures{Ligatures=TeX,Scale=MatchLowercase}
\fi
% use upquote if available, for straight quotes in verbatim environments
\IfFileExists{upquote.sty}{\usepackage{upquote}}{}
% use microtype if available
\IfFileExists{microtype.sty}{%
\usepackage[]{microtype}
\UseMicrotypeSet[protrusion]{basicmath} % disable protrusion for tt fonts
}{}
\IfFileExists{parskip.sty}{%
\usepackage{parskip}
}{% else
\setlength{\parindent}{0pt}
\setlength{\parskip}{6pt plus 2pt minus 1pt}
}
\usepackage{hyperref}
\hypersetup{
            pdftitle={Project\_Mollusca},
            pdfauthor={just\_student},
            pdfborder={0 0 0},
            breaklinks=true}
\urlstyle{same}  % don't use monospace font for urls
\usepackage[margin=1in]{geometry}
\usepackage{color}
\usepackage{fancyvrb}
\newcommand{\VerbBar}{|}
\newcommand{\VERB}{\Verb[commandchars=\\\{\}]}
\DefineVerbatimEnvironment{Highlighting}{Verbatim}{commandchars=\\\{\}}
% Add ',fontsize=\small' for more characters per line
\usepackage{framed}
\definecolor{shadecolor}{RGB}{248,248,248}
\newenvironment{Shaded}{\begin{snugshade}}{\end{snugshade}}
\newcommand{\AlertTok}[1]{\textcolor[rgb]{0.94,0.16,0.16}{#1}}
\newcommand{\AnnotationTok}[1]{\textcolor[rgb]{0.56,0.35,0.01}{\textbf{\textit{#1}}}}
\newcommand{\AttributeTok}[1]{\textcolor[rgb]{0.77,0.63,0.00}{#1}}
\newcommand{\BaseNTok}[1]{\textcolor[rgb]{0.00,0.00,0.81}{#1}}
\newcommand{\BuiltInTok}[1]{#1}
\newcommand{\CharTok}[1]{\textcolor[rgb]{0.31,0.60,0.02}{#1}}
\newcommand{\CommentTok}[1]{\textcolor[rgb]{0.56,0.35,0.01}{\textit{#1}}}
\newcommand{\CommentVarTok}[1]{\textcolor[rgb]{0.56,0.35,0.01}{\textbf{\textit{#1}}}}
\newcommand{\ConstantTok}[1]{\textcolor[rgb]{0.00,0.00,0.00}{#1}}
\newcommand{\ControlFlowTok}[1]{\textcolor[rgb]{0.13,0.29,0.53}{\textbf{#1}}}
\newcommand{\DataTypeTok}[1]{\textcolor[rgb]{0.13,0.29,0.53}{#1}}
\newcommand{\DecValTok}[1]{\textcolor[rgb]{0.00,0.00,0.81}{#1}}
\newcommand{\DocumentationTok}[1]{\textcolor[rgb]{0.56,0.35,0.01}{\textbf{\textit{#1}}}}
\newcommand{\ErrorTok}[1]{\textcolor[rgb]{0.64,0.00,0.00}{\textbf{#1}}}
\newcommand{\ExtensionTok}[1]{#1}
\newcommand{\FloatTok}[1]{\textcolor[rgb]{0.00,0.00,0.81}{#1}}
\newcommand{\FunctionTok}[1]{\textcolor[rgb]{0.00,0.00,0.00}{#1}}
\newcommand{\ImportTok}[1]{#1}
\newcommand{\InformationTok}[1]{\textcolor[rgb]{0.56,0.35,0.01}{\textbf{\textit{#1}}}}
\newcommand{\KeywordTok}[1]{\textcolor[rgb]{0.13,0.29,0.53}{\textbf{#1}}}
\newcommand{\NormalTok}[1]{#1}
\newcommand{\OperatorTok}[1]{\textcolor[rgb]{0.81,0.36,0.00}{\textbf{#1}}}
\newcommand{\OtherTok}[1]{\textcolor[rgb]{0.56,0.35,0.01}{#1}}
\newcommand{\PreprocessorTok}[1]{\textcolor[rgb]{0.56,0.35,0.01}{\textit{#1}}}
\newcommand{\RegionMarkerTok}[1]{#1}
\newcommand{\SpecialCharTok}[1]{\textcolor[rgb]{0.00,0.00,0.00}{#1}}
\newcommand{\SpecialStringTok}[1]{\textcolor[rgb]{0.31,0.60,0.02}{#1}}
\newcommand{\StringTok}[1]{\textcolor[rgb]{0.31,0.60,0.02}{#1}}
\newcommand{\VariableTok}[1]{\textcolor[rgb]{0.00,0.00,0.00}{#1}}
\newcommand{\VerbatimStringTok}[1]{\textcolor[rgb]{0.31,0.60,0.02}{#1}}
\newcommand{\WarningTok}[1]{\textcolor[rgb]{0.56,0.35,0.01}{\textbf{\textit{#1}}}}
\usepackage{graphicx,grffile}
\makeatletter
\def\maxwidth{\ifdim\Gin@nat@width>\linewidth\linewidth\else\Gin@nat@width\fi}
\def\maxheight{\ifdim\Gin@nat@height>\textheight\textheight\else\Gin@nat@height\fi}
\makeatother
% Scale images if necessary, so that they will not overflow the page
% margins by default, and it is still possible to overwrite the defaults
% using explicit options in \includegraphics[width, height, ...]{}
\setkeys{Gin}{width=\maxwidth,height=\maxheight,keepaspectratio}
\setlength{\emergencystretch}{3em}  % prevent overfull lines
\providecommand{\tightlist}{%
  \setlength{\itemsep}{0pt}\setlength{\parskip}{0pt}}
\setcounter{secnumdepth}{0}
% Redefines (sub)paragraphs to behave more like sections
\ifx\paragraph\undefined\else
\let\oldparagraph\paragraph
\renewcommand{\paragraph}[1]{\oldparagraph{#1}\mbox{}}
\fi
\ifx\subparagraph\undefined\else
\let\oldsubparagraph\subparagraph
\renewcommand{\subparagraph}[1]{\oldsubparagraph{#1}\mbox{}}
\fi

% set default figure placement to htbp
\makeatletter
\def\fps@figure{htbp}
\makeatother


\title{Project\_Mollusca}
\author{just\_student}
\date{10/8/2020}

\begin{document}
\maketitle

\hypertarget{ux43eux446ux435ux43dux43aux430-ux43fux430ux440ux430ux43cux435ux442ux440ux43eux432-ux440ux430ux437ux43cux435ux440ux430-ux438-ux432ux435ux441ux430-ux443-ux43cux43eux43bux43bux44eux441ux43aux43eux432-ux440ux430ux437ux43dux43eux433ux43e-ux43fux43eux43bux430-ux438-ux432ux43eux437ux440ux430ux441ux442ux430}{%
\section{Оценка параметров размера и веса у моллюсков разного пола и
возраста}\label{ux43eux446ux435ux43dux43aux430-ux43fux430ux440ux430ux43cux435ux442ux440ux43eux432-ux440ux430ux437ux43cux435ux440ux430-ux438-ux432ux435ux441ux430-ux443-ux43cux43eux43bux43bux44eux441ux43aux43eux432-ux440ux430ux437ux43dux43eux433ux43e-ux43fux43eux43bux430-ux438-ux432ux43eux437ux440ux430ux441ux442ux430}}

Необходимые нам библиотеки:

\begin{Shaded}
\begin{Highlighting}[]
\KeywordTok{options}\NormalTok{(}\DataTypeTok{encoding =} \StringTok{"UTF-8"}\NormalTok{)}

\NormalTok{PkgNames <-}\StringTok{ }\KeywordTok{c}\NormalTok{(}\StringTok{"dplyr"}\NormalTok{, }\StringTok{"magrittr"}\NormalTok{, }\StringTok{"ggplot2"}\NormalTok{, }\StringTok{"reshape2"}\NormalTok{, }\StringTok{"purrr"}\NormalTok{,}\StringTok{"utf8"}\NormalTok{, }\StringTok{"outliers"}\NormalTok{, }\StringTok{"DescTools"}\NormalTok{)}
\NormalTok{new.packages <-}\StringTok{ }\NormalTok{PkgNames[}\OperatorTok{!}\NormalTok{(PkgNames }\OperatorTok\StringTok{ }\KeywordTok{installed.packages}\NormalTok{()[,}\StringTok{"Package"}\NormalTok{])]}
\ControlFlowTok{if}\NormalTok{(}\KeywordTok{length}\NormalTok{(new.packages)) }\KeywordTok{install.packages}\NormalTok{(new.packages)}

\KeywordTok{invisible}\NormalTok{(}\KeywordTok{suppressMessages}\NormalTok{(}\KeywordTok{suppressWarnings}\NormalTok{(}\KeywordTok{lapply}\NormalTok{(PkgNames, require, }\DataTypeTok{character.only =}\NormalTok{ T))))}
\end{Highlighting}
\end{Shaded}

\hypertarget{ux43aux440ux430ux442ux43aux43eux435-ux43eux43fux438ux441ux430ux43dux438ux435-ux43fux440ux43eux435ux43aux442ux430}{%
\subsection{Краткое описание
проекта}\label{ux43aux440ux430ux442ux43aux43eux435-ux43eux43fux438ux441ux430ux43dux438ux435-ux43fux440ux43eux435ux43aux442ux430}}

Цель исследования: данный проект представляет собой попытку
проанализировать данные, характеризующиее выборку беломорских мидий и
описывающие их размер, вес, пол и возраст.

Задачи:

\begin{enumerate}
\def\labelenumi{\arabic{enumi}.}
\tightlist
\item
  Провести первичный анализ данных на предмет ошибок,пропущенных и
  некорректных значений.
\item
  Оценить параметры распределения переменных, характеризующих вес и
  размер моллюска, и выявить коррелирующие признаки.
\item
  Оценить влияние возраста и пола моллюска на параметры размера и веса
\item
  Попытаться выявить наличие взаимосвязи между переменными,
  характеризующими размер и вес моллюсков.
\end{enumerate}

\hypertarget{dataframe-generation}{%
\subsubsection{Dataframe generation}\label{dataframe-generation}}

Соберем полную таблицу с данными из исходных файлов и выведем структуру
получившегося объекта. Как мы видим, в получившемся датасете последняя и
первые три переменные имеют строковый тип, оставшиеся - нумерический.
Общее количество наблюдений по всем группам - 4177.

\begin{Shaded}
\begin{Highlighting}[]
\NormalTok{abs_way <-}\StringTok{ }\KeywordTok{getwd}\NormalTok{() }
\NormalTok{abs_way <-}\StringTok{ }\KeywordTok{normalizePath}\NormalTok{(abs_way)}
\NormalTok{data_files<-}\StringTok{ }\KeywordTok{list.files}\NormalTok{(abs_way, }\DataTypeTok{pattern =}\StringTok{".csv"}\NormalTok{)}
\NormalTok{Mollusca <-}\StringTok{ }\KeywordTok{list}\NormalTok{()}
\ControlFlowTok{for}\NormalTok{(i }\ControlFlowTok{in} \DecValTok{1}\OperatorTok{:}\KeywordTok{length}\NormalTok{(data_files))\{}
\NormalTok{file <-}\StringTok{ }\KeywordTok{read.table}\NormalTok{(data_files[i], }\DataTypeTok{header=}\OtherTok{TRUE}\NormalTok{, }\DataTypeTok{sep=}\StringTok{","}\NormalTok{, }\DataTypeTok{stringsAsFactors=}\OtherTok{TRUE}\NormalTok{)}
\NormalTok{Mollusca <-}\StringTok{ }\KeywordTok{rbind}\NormalTok{(Mollusca, file)}
\NormalTok{\}}
\KeywordTok{str}\NormalTok{(Mollusca)}
\end{Highlighting}
\end{Shaded}

\begin{verbatim}
## 'data.frame':    4177 obs. of  9 variables:
##  $ Rings                                 : chr  "18" "14" "8" "12" ...
##  $ Sex..1...male..2...female..3...uvenil.: chr  "2" "3" "1" "1" ...
##  $ Length                                : chr  "0.575" "0.385" "0.475" "0.665" ...
##  $ Diameter                              : num  0.45 0.305 0.37 0.525 0.12 0.16 0.565 0.43 0.175 0.3 ...
##  $ Height                                : num  0.17 0.095 0.125 0.18 0.075 0.05 0.2 0.15 0.055 0.115 ...
##  $ Whole_weight                          : num  1.048 0.252 0.649 1.429 0.117 ...
##  $ Shucked_weight                        : num  0.3775 0.0915 0.347 0.6715 0.0455 ...
##  $ Viscera_weight                        : num  0.171 0.055 0.136 0.29 0.029 ...
##  $ Shell_weight                          : num  0.385 0.09 0.142 0.4 0.0345 0.015 0.494 0.221 0.018 0.0935 ...
\end{verbatim}

\hypertarget{ux440ux430ux431ux43eux442ux430-ux441-ux43fux435ux440ux435ux43cux435ux43dux43dux44bux43cux438}{%
\subsubsection{Работа с
переменными}\label{ux440ux430ux431ux43eux442ux430-ux441-ux43fux435ux440ux435ux43cux435ux43dux43dux44bux43cux438}}

Переименуем вторую переменную нашего датафрейма
``Sex..1\ldots{}male..2\ldots{}female..3\ldots{}uvenil.'' в переменную
`Sex'.

\begin{Shaded}
\begin{Highlighting}[]
\KeywordTok{colnames}\NormalTok{(Mollusca)[}\KeywordTok{colnames}\NormalTok{(Mollusca) }\OperatorTok{==}\StringTok{ 'Sex..1...male..2...female..3...uvenil.'}\NormalTok{] <-}\StringTok{ 'Sex'}
\KeywordTok{colnames}\NormalTok{(Mollusca)}
\end{Highlighting}
\end{Shaded}

\begin{verbatim}
## [1] "Rings"          "Sex"            "Length"         "Diameter"      
## [5] "Height"         "Whole_weight"   "Shucked_weight" "Viscera_weight"
## [9] "Shell_weight"
\end{verbatim}

Сначала проверим все колонки нумерического типа на наличие пропущенных
значений (NA). Как мы видим, их довольно мало, всего 22.

\begin{Shaded}
\begin{Highlighting}[]
\NormalTok{Mollusca }\OperatorTok\StringTok{ }\KeywordTok{select_all}\NormalTok{() }\OperatorTok\KeywordTok{summarise_all}\NormalTok{(}\KeywordTok{funs}\NormalTok{(}\KeywordTok{sum}\NormalTok{(}\KeywordTok{is.na}\NormalTok{(.))))}
\end{Highlighting}
\end{Shaded}

\begin{verbatim}
##   Rings Sex Length Diameter Height Whole_weight Shucked_weight Viscera_weight
## 1     0   1      6        5      2            1              3              0
##   Shell_weight
## 1            2
\end{verbatim}

Далее, изменим тип переменной Diameter на numeric. Т.к.в нашем датасете
есть еще 2 переменные строкового типа, а именно Lenght и Rings, то
проверим их на уникальные элементы.

\begin{Shaded}
\begin{Highlighting}[]
\NormalTok{Mollusca}\OperatorTok{$}\NormalTok{Diameter <-}\StringTok{  }\KeywordTok{as.numeric}\NormalTok{(Mollusca}\OperatorTok{$}\NormalTok{Diameter,}\DataTypeTok{quietly =} \OtherTok{TRUE}\NormalTok{) }\CommentTok{#convert  in numeric type}
\NormalTok{Rings <-}\StringTok{ }\KeywordTok{unique}\NormalTok{(Mollusca}\OperatorTok{$}\NormalTok{Rings)}
\KeywordTok{print}\NormalTok{(Rings[}\DecValTok{29}\NormalTok{])}
\end{Highlighting}
\end{Shaded}

\begin{verbatim}
## [1] "nine"
\end{verbatim}

\begin{Shaded}
\begin{Highlighting}[]
\NormalTok{Length <-}\StringTok{ }\KeywordTok{unique}\NormalTok{(Mollusca}\OperatorTok{$}\NormalTok{Length)}
\KeywordTok{print}\NormalTok{(Length[}\DecValTok{134}\NormalTok{])}
\end{Highlighting}
\end{Shaded}

\begin{verbatim}
## [1] "No data! I forgot to mesure it!("
\end{verbatim}

Как мы видим, в интересующих нас колонках обнаруживаются некорректные
наблюдения. Заменим значения этих наблюдений на корректные, либо на NA.
Проверим наши данные на количество пропущенных значений, оно увеличилось
на 1.

\begin{Shaded}
\begin{Highlighting}[]
\NormalTok{Mollusca}\OperatorTok{$}\NormalTok{Rings[Mollusca}\OperatorTok{$}\NormalTok{Rings }\OperatorTok{==}\StringTok{ "nine"}\NormalTok{] <-}\StringTok{ "9"}
\NormalTok{Mollusca}\OperatorTok{$}\NormalTok{Length[Mollusca}\OperatorTok{$}\NormalTok{Length }\OperatorTok{==}\StringTok{ "No data! I forgot to mesure it!("}\NormalTok{] <-}\StringTok{ }\OtherTok{NA}

\NormalTok{Mollusca}\OperatorTok{$}\NormalTok{Rings <-}\StringTok{ }\KeywordTok{as.numeric}\NormalTok{(Mollusca}\OperatorTok{$}\NormalTok{Rings, }\DataTypeTok{quietly =} \OtherTok{TRUE}\NormalTok{)}
\NormalTok{Mollusca}\OperatorTok{$}\NormalTok{Length <-}\StringTok{  }\KeywordTok{as.double}\NormalTok{(Mollusca}\OperatorTok{$}\NormalTok{Length, }\DataTypeTok{queitly =} \OtherTok{TRUE}\NormalTok{)}
\NormalTok{Mollusca }\OperatorTok\StringTok{ }\KeywordTok{select_all}\NormalTok{() }\OperatorTok\StringTok{ }\KeywordTok{summarise_all}\NormalTok{(}\KeywordTok{funs}\NormalTok{(}\KeywordTok{sum}\NormalTok{(}\KeywordTok{is.na}\NormalTok{(.))))}
\end{Highlighting}
\end{Shaded}

\begin{verbatim}
##   Rings Sex Length Diameter Height Whole_weight Shucked_weight Viscera_weight
## 1     0   1      7        5      2            1              3              0
##   Shell_weight
## 1            2
\end{verbatim}

Найдем все уникальные значения для переменной Sex и выведем их. Как мы
видим, в этой переменной тоже присутствуют некорректные значения.

\begin{Shaded}
\begin{Highlighting}[]
\NormalTok{sex <-}\StringTok{ }\KeywordTok{unique}\NormalTok{(Mollusca}\OperatorTok{$}\NormalTok{Sex)}
\KeywordTok{print}\NormalTok{(sex)}\CommentTok{#check for unique values}
\end{Highlighting}
\end{Shaded}

\begin{verbatim}
## [1] "2"     "3"     "1"     NA      "three" "one"   "male"
\end{verbatim}

Как и в предыдущем случае, у нас есть ошибки в данных, исправим их на
корректный формат. Затем сконвертируем переменную Sex в фактор с тремя
градациями: Male, Female, Uvenile.

\begin{Shaded}
\begin{Highlighting}[]
\NormalTok{Mollusca}\OperatorTok{$}\NormalTok{Sex[Mollusca}\OperatorTok{$}\NormalTok{Sex }\OperatorTok{==}\StringTok{ "three"}\NormalTok{] <-}\StringTok{ "3"} 
\NormalTok{Mollusca}\OperatorTok{$}\NormalTok{Sex[Mollusca}\OperatorTok{$}\NormalTok{Sex }\OperatorTok{==}\StringTok{ "one"}\NormalTok{] <-}\StringTok{ "1"}
\NormalTok{Mollusca}\OperatorTok{$}\NormalTok{Sex[Mollusca}\OperatorTok{$}\NormalTok{Sex }\OperatorTok{==}\StringTok{ "male"}\NormalTok{] <-}\StringTok{ "1"}
\NormalTok{sex_new <-}\StringTok{ }\KeywordTok{unique}\NormalTok{(Mollusca}\OperatorTok{$}\NormalTok{Sex) }\CommentTok{#after changing the uncorrect values}
\NormalTok{Mollusca}\OperatorTok{$}\NormalTok{Sex <-}\StringTok{ }\KeywordTok{factor}\NormalTok{(Mollusca}\OperatorTok{$}\NormalTok{Sex, }\DataTypeTok{levels =} \KeywordTok{c}\NormalTok{(}\StringTok{"1"}\NormalTok{, }\StringTok{"2"}\NormalTok{,}\StringTok{"3"}\NormalTok{), }\DataTypeTok{labels =} \KeywordTok{c}\NormalTok{(}\StringTok{"Male"}\NormalTok{, }\StringTok{"Female"}\NormalTok{, }\StringTok{"Uvenile"}\NormalTok{))}
\KeywordTok{str}\NormalTok{(Mollusca)}
\end{Highlighting}
\end{Shaded}

\begin{verbatim}
## 'data.frame':    4177 obs. of  9 variables:
##  $ Rings         : num  18 14 8 12 4 4 12 8 6 8 ...
##  $ Sex           : Factor w/ 3 levels "Male","Female",..: 2 3 1 1 3 3 1 2 3 2 ...
##  $ Length        : num  0.575 0.385 0.475 0.665 0.28 0.22 0.72 0.55 0.235 0.4 ...
##  $ Diameter      : num  0.45 0.305 0.37 0.525 0.12 0.16 0.565 0.43 0.175 0.3 ...
##  $ Height        : num  0.17 0.095 0.125 0.18 0.075 0.05 0.2 0.15 0.055 0.115 ...
##  $ Whole_weight  : num  1.048 0.252 0.649 1.429 0.117 ...
##  $ Shucked_weight: num  0.3775 0.0915 0.347 0.6715 0.0455 ...
##  $ Viscera_weight: num  0.171 0.055 0.136 0.29 0.029 ...
##  $ Shell_weight  : num  0.385 0.09 0.142 0.4 0.0345 0.015 0.494 0.221 0.018 0.0935 ...
\end{verbatim}

\hypertarget{ux438ux434ux435ux43dux442ux438ux444ux438ux43aux430ux446ux438ux44f-ux432ux44bux431ux440ux43eux441ux43eux432}{%
\subsubsection{Идентификация
выбросов}\label{ux438ux434ux435ux43dux442ux438ux444ux438ux43aux430ux446ux438ux44f-ux432ux44bux431ux440ux43eux441ux43eux432}}

Идентифицируем все выбросы в наших данных и создадим на их основе список
List\_of\_outl. Как мы видим из структуры списка, наибольшее количество
выбросов приходится на переменную Rings. Найдем все уникальные значения
выбросов в Rings и выведем их. Очевидно, что туда попадают совсем юные
(до 3 лет) и старые животные (больше 16 лет). Не будем выбрасывать эти
данные совсем, лучше сформируем из них отдельные группы.

\begin{Shaded}
\begin{Highlighting}[]
\NormalTok{outliers <-}\StringTok{ }\ControlFlowTok{function}\NormalTok{(dataframe)\{}
\NormalTok{  dataframe }\OperatorTok
\StringTok{      }\KeywordTok{select_if}\NormalTok{(is.numeric) }\OperatorTok\StringTok{ }
\StringTok{      }\KeywordTok{map}\NormalTok{(}\OperatorTok{~}\StringTok{ }\KeywordTok{boxplot.stats}\NormalTok{(.x)}\OperatorTok{$}\NormalTok{out)}
\NormalTok{\}}
\NormalTok{List_of_outl <-}\StringTok{ }\KeywordTok{outliers}\NormalTok{(Mollusca)}
\KeywordTok{str}\NormalTok{(List_of_outl)}
\end{Highlighting}
\end{Shaded}

\begin{verbatim}
## List of 8
##  $ Rings         : num [1:278] 18 20 17 20 18 18 3 16 16 16 ...
##  $ Length        : num [1:49] 0.155 0.165 0.17 0.165 0.16 0.2 0.19 0.13 0.165 0.16 ...
##  $ Diameter      : num [1:58] 0.12 0.105 0.12 0.13 0.15 0.115 0.12 0.145 0.15 0.13 ...
##  $ Height        : num [1:29] 0.03 0.015 0.02 0.25 0.03 0.03 0.03 0 0.025 0.025 ...
##  $ Whole_weight  : num [1:30] 2.22 2.83 2.25 2.38 2.24 ...
##  $ Shucked_weight: num [1:48] 1.017 1.083 1.147 0.984 0.992 ...
##  $ Viscera_weight: num [1:26] 0.55 0.519 0.5 0.525 0.564 ...
##  $ Shell_weight  : num [1:35] 0.63 0.897 0.66 0.655 0.797 ...
\end{verbatim}

\begin{Shaded}
\begin{Highlighting}[]
\NormalTok{uniq_Rings_out <-}\StringTok{ }\KeywordTok{unique}\NormalTok{(List_of_outl}\OperatorTok{$}\NormalTok{Rings)}
\KeywordTok{print}\NormalTok{(uniq_Rings_out)}
\end{Highlighting}
\end{Shaded}

\begin{verbatim}
##  [1] 18 20 17  3 16 19 23 29 22 21  2 25 27 24  1 26
\end{verbatim}

\hypertarget{ux43eux446ux435ux43dux43aux430-ux43dux430ux43bux438ux447ux438ux44f-ux432ux44bux431ux440ux43eux441ux43eux432-ux432-ux43fux435ux440ux435ux43cux435ux43dux43dux44bux445-ux441-ux43fux43eux43cux43eux449ux44cux44e-ux433ux440ux430ux444ux438ux43aux430-ux431ux43eux43aux441ux43fux43bux43eux442}{%
\subsubsection{Оценка наличия выбросов в переменных с помощью графика
боксплот}\label{ux43eux446ux435ux43dux43aux430-ux43dux430ux43bux438ux447ux438ux44f-ux432ux44bux431ux440ux43eux441ux43eux432-ux432-ux43fux435ux440ux435ux43cux435ux43dux43dux44bux445-ux441-ux43fux43eux43cux43eux449ux44cux44e-ux433ux440ux430ux444ux438ux43aux430-ux431ux43eux43aux441ux43fux43bux43eux442}}

Построим графики распределения наших переменных в зависимости от пола.
Можно заметить, что значения всех параметров для группы ``Uvenile''
меньше, чем для половозрелых особей. Возможно, это связано с тем, что
ювенильные животные в целом моложе своих половозрелых сородичей.

\begin{Shaded}
\begin{Highlighting}[]
\NormalTok{size_weight =}\StringTok{ }\KeywordTok{melt}\NormalTok{(Mollusca, }\DataTypeTok{id.vars =} \StringTok{"Sex"}\NormalTok{, }\DataTypeTok{measure.vars =} \KeywordTok{c}\NormalTok{(}\StringTok{"Rings"}\NormalTok{,}\StringTok{"Length"}\NormalTok{, }\StringTok{"Diameter"}\NormalTok{, }\StringTok{"Height"}\NormalTok{, }\StringTok{"Whole_weight"}\NormalTok{, }\StringTok{"Shucked_weight"}\NormalTok{, }\StringTok{"Viscera_weight"}\NormalTok{, }\StringTok{"Shell_weight"}\NormalTok{), }\DataTypeTok{quietly =} \OtherTok{TRUE}\NormalTok{)}
\KeywordTok{ggplot}\NormalTok{(size_weight, }\KeywordTok{aes}\NormalTok{(}\DataTypeTok{x =}\NormalTok{ variable,}\DataTypeTok{y =}\NormalTok{ value, }\DataTypeTok{col =}\NormalTok{ Sex),}\DataTypeTok{quietly =} \OtherTok{TRUE}\NormalTok{) }\OperatorTok{+}\StringTok{ }\KeywordTok{facet_wrap}\NormalTok{(}\OperatorTok{~}\NormalTok{variable, }\DataTypeTok{scale=}\StringTok{"free"}\NormalTok{) }\OperatorTok{+}\StringTok{ }\KeywordTok{geom_boxplot}\NormalTok{( }\DataTypeTok{size =} \DecValTok{1}\NormalTok{)}\OperatorTok{+}\StringTok{ }\KeywordTok{labs}\NormalTok{( }\DataTypeTok{y =} \StringTok{"value"}\NormalTok{, }\DataTypeTok{x =} \OtherTok{NULL}\NormalTok{) }\OperatorTok{+}\StringTok{ }\KeywordTok{theme}\NormalTok{(}\DataTypeTok{axis.text.x =} \KeywordTok{element_text}\NormalTok{(}\DataTypeTok{angle=}\DecValTok{0}\NormalTok{, }\DataTypeTok{hjust=}\DecValTok{1}\NormalTok{, }\DataTypeTok{vjust=}\FloatTok{0.5}\NormalTok{))}\OperatorTok{+}\StringTok{ }\KeywordTok{theme}\NormalTok{(}\DataTypeTok{legend.position =} \StringTok{"left"}\NormalTok{)}
\end{Highlighting}
\end{Shaded}

\includegraphics{Project_files/figure-latex/unnamed-chunk-10-1.pdf}

Уберем выбросы во всех нумерических переменных, кроме Rings. Для этого
заменим их на пропущенные значения, которые затем удалим.

\begin{Shaded}
\begin{Highlighting}[]
\NormalTok{ Mollusca_clean <-}\StringTok{ }\NormalTok{Mollusca[,}\DecValTok{3}\OperatorTok{:}\DecValTok{9}\NormalTok{] }\OperatorTok\StringTok{          }
\StringTok{           }\KeywordTok{map_if}\NormalTok{(is.numeric, }\OperatorTok{~}\StringTok{ }\KeywordTok{replace}\NormalTok{(.x, .x }\OperatorTok\StringTok{ }\KeywordTok{boxplot.stats}\NormalTok{(.x)}\OperatorTok{$}\NormalTok{out, }\OtherTok{NA}\NormalTok{)) }\OperatorTok
\StringTok{           }\NormalTok{bind_cols }
\NormalTok{Mollusca }\OperatorTok\StringTok{ }\KeywordTok{summarise_all}\NormalTok{(}\KeywordTok{funs}\NormalTok{( }\KeywordTok{sum}\NormalTok{(}\KeywordTok{is.na}\NormalTok{(.))))}
\end{Highlighting}
\end{Shaded}

\begin{verbatim}
##   Rings Sex Length Diameter Height Whole_weight Shucked_weight Viscera_weight
## 1     0   1      7        5      2            1              3              0
##   Shell_weight
## 1            2
\end{verbatim}

\begin{Shaded}
\begin{Highlighting}[]
\NormalTok{Mollusca <-}\StringTok{ }\NormalTok{Mollusca[}\KeywordTok{complete.cases}\NormalTok{(Mollusca_clean),]}
\end{Highlighting}
\end{Shaded}

\hypertarget{ux43eux446ux435ux43dux43aux430-ux43dux43eux440ux43cux430ux43bux44cux43dux43eux441ux442ux438-ux440ux430ux441ux43fux440ux435ux434ux435ux43bux435ux43dux438ux44f-ux434ux430ux43dux43dux44bux445}{%
\subsubsection{Оценка нормальности распределения
данных}\label{ux43eux446ux435ux43dux43aux430-ux43dux43eux440ux43cux430ux43bux44cux43dux43eux441ux442ux438-ux440ux430ux441ux43fux440ux435ux434ux435ux43bux435ux43dux438ux44f-ux434ux430ux43dux43dux44bux445}}

Проверим все наши нумерические переменные на нормальность распределения
и построим их гистограммы. Для проверки на нормальность будем
использовать тест Шапиро-Уилка. На графиках отобразим распределения
признаков, сгруппированных по полу и возрасту. Как видно из таблицы и
графика, данные имеют ненормальное распределение (Ha), поэтому для
дальнейшего анализа необходимо использовать непараметрические критерии.

\begin{Shaded}
\begin{Highlighting}[]
\NormalTok{shapiro_test_df <-}\StringTok{ }\ControlFlowTok{function}\NormalTok{(df, }\DataTypeTok{bonf=} \OtherTok{FALSE}\NormalTok{, }\DataTypeTok{alpha=} \FloatTok{0.05}\NormalTok{) \{}
\NormalTok{        l <-}\StringTok{ }\KeywordTok{lapply}\NormalTok{(df, shapiro.test)}
\NormalTok{        s <-}\StringTok{ }\KeywordTok{do.call}\NormalTok{(}\StringTok{"c"}\NormalTok{, }\KeywordTok{lapply}\NormalTok{(l, }\StringTok{"[["}\NormalTok{, }\DecValTok{1}\NormalTok{))}
\NormalTok{        p <-}\StringTok{ }\KeywordTok{do.call}\NormalTok{(}\StringTok{"c"}\NormalTok{, }\KeywordTok{lapply}\NormalTok{(l, }\StringTok{"[["}\NormalTok{, }\DecValTok{2}\NormalTok{))}
        \ControlFlowTok{if}\NormalTok{ (bonf }\OperatorTok{==}\StringTok{ }\OtherTok{TRUE}\NormalTok{) \{}
\NormalTok{                sig <-}\StringTok{ }\KeywordTok{ifelse}\NormalTok{(p }\OperatorTok{>}\StringTok{ }\NormalTok{alpha }\OperatorTok{/}\StringTok{ }\KeywordTok{length}\NormalTok{(l), }\StringTok{"H0"}\NormalTok{, }\StringTok{"Ha"}\NormalTok{)}
\NormalTok{        \} }\ControlFlowTok{else}\NormalTok{ \{}
\NormalTok{                sig <-}\StringTok{ }\KeywordTok{ifelse}\NormalTok{(p }\OperatorTok{>}\StringTok{ }\NormalTok{alpha, }\StringTok{"H0"}\NormalTok{, }\StringTok{"Ha"}\NormalTok{)}
\NormalTok{        \}}
        \KeywordTok{return}\NormalTok{(}\KeywordTok{list}\NormalTok{(}\DataTypeTok{statistic=}\NormalTok{ s,}
                    \DataTypeTok{p.value=}\NormalTok{ p,}
                    \DataTypeTok{significance=}\NormalTok{ sig,}
                    \DataTypeTok{method=} \KeywordTok{ifelse}\NormalTok{(bonf }\OperatorTok{==}\StringTok{ }\OtherTok{TRUE}\NormalTok{, }\StringTok{"Shapiro-Wilks test with Bonferroni Correction"}\NormalTok{,}
                                   \StringTok{"Shapiro-Wilks test without Bonferroni Correction"}\NormalTok{)))}
\NormalTok{\}}

\NormalTok{Mollusca  }\OperatorTok\StringTok{ }\KeywordTok{select}\NormalTok{(}\KeywordTok{where}\NormalTok{(is.numeric)) }\OperatorTok\StringTok{ }\KeywordTok{shapiro_test_df}\NormalTok{()}
\end{Highlighting}
\end{Shaded}

\begin{verbatim}
## $statistic
##          Rings.W         Length.W       Diameter.W         Height.W 
##        0.9209510        0.9692844        0.9693539        0.9909639 
##   Whole_weight.W Shucked_weight.W Viscera_weight.W   Shell_weight.W 
##        0.9761246        0.9723527        0.9724933        0.9783441 
## 
## $p.value
##          Rings         Length       Diameter         Height   Whole_weight 
##   1.234612e-41   1.840449e-28   1.965048e-28   2.737685e-15   2.158914e-25 
## Shucked_weight Viscera_weight   Shell_weight 
##   3.713060e-27   4.286780e-27   2.972915e-24 
## 
## $significance
##          Rings         Length       Diameter         Height   Whole_weight 
##           "Ha"           "Ha"           "Ha"           "Ha"           "Ha" 
## Shucked_weight Viscera_weight   Shell_weight 
##           "Ha"           "Ha"           "Ha" 
## 
## $method
## [1] "Shapiro-Wilks test without Bonferroni Correction"
\end{verbatim}

\begin{Shaded}
\begin{Highlighting}[]
\NormalTok{size_weight <-}\StringTok{ }\KeywordTok{melt}\NormalTok{(Mollusca, }\DataTypeTok{id.vars =} \KeywordTok{c}\NormalTok{(}\StringTok{"Sex"}\NormalTok{,}\StringTok{"Age"}\NormalTok{), }\DataTypeTok{measure.vars =} \KeywordTok{c}\NormalTok{(}\StringTok{"Rings"}\NormalTok{,}\StringTok{"Length"}\NormalTok{, }\StringTok{"Diameter"}\NormalTok{, }\StringTok{"Height"}\NormalTok{, }\StringTok{"Whole_weight"}\NormalTok{, }\StringTok{"Shucked_weight"}\NormalTok{, }\StringTok{"Viscera_weight"}\NormalTok{, }\StringTok{"Shell_weight"}\NormalTok{), }\DataTypeTok{quietly =} \OtherTok{TRUE}\NormalTok{)}

\KeywordTok{ggplot}\NormalTok{(size_weight, }\KeywordTok{aes}\NormalTok{(}\DataTypeTok{x =}\NormalTok{ value, }\DataTypeTok{col=}\NormalTok{Sex),}\DataTypeTok{quietly =} \OtherTok{TRUE}\NormalTok{) }\OperatorTok{+}\StringTok{ }\KeywordTok{facet_wrap}\NormalTok{(}\OperatorTok{~}\NormalTok{variable, }\DataTypeTok{scale=}\StringTok{"free"}\NormalTok{) }\OperatorTok{+}\StringTok{ }\KeywordTok{geom_histogram}\NormalTok{(}\DataTypeTok{size =} \DecValTok{1}\NormalTok{,}\DataTypeTok{stat =} \StringTok{"count"}\NormalTok{)}\OperatorTok{+}\StringTok{ }\KeywordTok{labs}\NormalTok{( }\DataTypeTok{y =} \StringTok{"value"}\NormalTok{, }\DataTypeTok{x =} \OtherTok{NULL}\NormalTok{) }\OperatorTok{+}\StringTok{ }\KeywordTok{theme}\NormalTok{(}\DataTypeTok{axis.text.x =} \KeywordTok{element_text}\NormalTok{(}\DataTypeTok{angle=}\DecValTok{0}\NormalTok{, }\DataTypeTok{hjust=}\DecValTok{1}\NormalTok{, }\DataTypeTok{vjust=}\FloatTok{0.5}\NormalTok{))}\OperatorTok{+}\StringTok{ }\KeywordTok{theme}\NormalTok{(}\DataTypeTok{legend.position =} \StringTok{"bottom"}\NormalTok{)}\OperatorTok{+}\KeywordTok{labs}\NormalTok{(}\DataTypeTok{title =} \StringTok{"Distributions of numeric variables across age"}\NormalTok{)}
\end{Highlighting}
\end{Shaded}

\includegraphics{Project_files/figure-latex/unnamed-chunk-14-1.pdf}

\begin{Shaded}
\begin{Highlighting}[]
\KeywordTok{ggplot}\NormalTok{(size_weight, }\KeywordTok{aes}\NormalTok{(}\DataTypeTok{x =}\NormalTok{ value, }\DataTypeTok{col=}\NormalTok{Age),}\DataTypeTok{quietly =} \OtherTok{TRUE}\NormalTok{) }\OperatorTok{+}\StringTok{ }\KeywordTok{facet_wrap}\NormalTok{(}\OperatorTok{~}\NormalTok{variable, }\DataTypeTok{scale=}\StringTok{"free"}\NormalTok{) }\OperatorTok{+}\StringTok{ }\KeywordTok{geom_histogram}\NormalTok{(}\DataTypeTok{size =} \DecValTok{1}\NormalTok{,}\DataTypeTok{stat =} \StringTok{"count"}\NormalTok{)}\OperatorTok{+}\StringTok{ }\KeywordTok{labs}\NormalTok{( }\DataTypeTok{y =} \StringTok{"value"}\NormalTok{, }\DataTypeTok{x =} \OtherTok{NULL}\NormalTok{) }\OperatorTok{+}\StringTok{ }\KeywordTok{theme}\NormalTok{(}\DataTypeTok{axis.text.x =} \KeywordTok{element_text}\NormalTok{(}\DataTypeTok{angle=}\DecValTok{0}\NormalTok{, }\DataTypeTok{hjust=}\DecValTok{1}\NormalTok{, }\DataTypeTok{vjust=}\FloatTok{0.5}\NormalTok{))}\OperatorTok{+}\StringTok{ }\KeywordTok{theme}\NormalTok{(}\DataTypeTok{legend.position =} \StringTok{"bottom"}\NormalTok{)}\OperatorTok{+}\KeywordTok{labs}\NormalTok{(}\DataTypeTok{title =} \StringTok{"Distributions of numeric variables across age"}\NormalTok{)}
\end{Highlighting}
\end{Shaded}

\hypertarget{section}{%
\subsection[]{\texorpdfstring{\protect\includegraphics{Project_files/figure-latex/unnamed-chunk-14-2.pdf}}{}}\label{section}}

\hypertarget{ux440ux430ux441ux447ux435ux442-ux441ux442ux430ux442ux438ux441ux442ux438ux447ux435ux441ux43aux438ux445-ux43fux43eux43aux430ux437ux430ux442ux435ux43bux435ux439-ux440ux430ux441ux43fux440ux435ux434ux435ux43bux435ux43dux438ux44f-ux43fux435ux440ux435ux43cux435ux43dux43dux44bux445-ux432-ux437ux430ux432ux438ux441ux438ux43cux43eux441ux442ux438-ux43eux442-ux43fux43eux43bux430-ux438-ux432ux43eux437ux440ux430ux441ux442ux430-ux43cux43eux43bux43bux44eux441ux43aux43eux432}{%
\subsubsection{Расчет статистических показателей распределения
переменных в зависимости от пола и возраста
моллюсков}\label{ux440ux430ux441ux447ux435ux442-ux441ux442ux430ux442ux438ux441ux442ux438ux447ux435ux441ux43aux438ux445-ux43fux43eux43aux430ux437ux430ux442ux435ux43bux435ux439-ux440ux430ux441ux43fux440ux435ux434ux435ux43bux435ux43dux438ux44f-ux43fux435ux440ux435ux43cux435ux43dux43dux44bux445-ux432-ux437ux430ux432ux438ux441ux438ux43cux43eux441ux442ux438-ux43eux442-ux43fux43eux43bux430-ux438-ux432ux43eux437ux440ux430ux441ux442ux430-ux43cux43eux43bux43bux44eux441ux43aux43eux432}}

Расчитаем статистические показатели распределения для переменных
Diameter и Whole\_weight для групп животных разного пола и возраста.

\begin{Shaded}
\begin{Highlighting}[]
\NormalTok{Diam <-}\StringTok{ }\KeywordTok{aggregate}\NormalTok{(Diameter }\OperatorTok{~}\StringTok{ }\NormalTok{Sex }\OperatorTok{+}\StringTok{ }\NormalTok{Age, }\DataTypeTok{data =}\NormalTok{ Mollusca, }\ControlFlowTok{function}\NormalTok{(x) }\KeywordTok{c}\NormalTok{ (}\DataTypeTok{mean =} \KeywordTok{mean}\NormalTok{(x),}\DataTypeTok{median =} \KeywordTok{median}\NormalTok{(x),}\DataTypeTok{sd =} \KeywordTok{sd}\NormalTok{(x),}\DataTypeTok{first_quant =} \KeywordTok{quantile}\NormalTok{(x,}\FloatTok{0.25}\NormalTok{), }\DataTypeTok{third_quant =} \KeywordTok{quantile}\NormalTok{(x,}\FloatTok{0.75}\NormalTok{)))}
\NormalTok{Wheight <-}\StringTok{ }\KeywordTok{aggregate}\NormalTok{(Whole_weight }\OperatorTok{~}\StringTok{ }\NormalTok{Sex }\OperatorTok{+}\StringTok{ }\NormalTok{Age, }\DataTypeTok{data =}\NormalTok{ Mollusca, }\ControlFlowTok{function}\NormalTok{(x) }\KeywordTok{c}\NormalTok{ (}\DataTypeTok{mean =} \KeywordTok{mean}\NormalTok{(x),}\DataTypeTok{median =} \KeywordTok{median}\NormalTok{(x),}\DataTypeTok{sd =} \KeywordTok{sd}\NormalTok{(x),}\DataTypeTok{first_quant =} \KeywordTok{quantile}\NormalTok{(x,}\FloatTok{0.25}\NormalTok{),}\DataTypeTok{third_quant =} \KeywordTok{quantile}\NormalTok{(x,}\FloatTok{0.75}\NormalTok{)))}

\NormalTok{Sum_Diam <-}\StringTok{ }\KeywordTok{cbind}\NormalTok{(Diam[}\OperatorTok{-}\KeywordTok{ncol}\NormalTok{(Diam)], Diam[[}\KeywordTok{ncol}\NormalTok{(Diam)]])}
\NormalTok{Sum_Wheight <-}\StringTok{ }\KeywordTok{cbind}\NormalTok{(Wheight[}\OperatorTok{-}\KeywordTok{ncol}\NormalTok{(Wheight)], Wheight[[}\KeywordTok{ncol}\NormalTok{(Wheight)]])}
\KeywordTok{print}\NormalTok{(Sum_Diam)}
\end{Highlighting}
\end{Shaded}

\begin{verbatim}
##        Sex   Age      mean median         sd first_quant.25% third_quant.75%
## 1     Male  baby 0.2081818 0.2050 0.03736795         0.19000         0.21250
## 2   Female  baby 0.2412500 0.2475 0.03816084         0.21750         0.27125
## 3  Uvenile  baby 0.2139655 0.2075 0.03980669         0.18375         0.23500
## 4     Male young 0.4207669 0.4400 0.07966582         0.37500         0.47500
## 5   Female young 0.4368482 0.4500 0.06890963         0.39500         0.48500
## 6  Uvenile young 0.3362760 0.3400 0.06938202         0.28000         0.38500
## 7     Male adult 0.4609145 0.4700 0.06567803         0.42000         0.51000
## 8   Female adult 0.4693016 0.4800 0.06359791         0.42500         0.51500
## 9  Uvenile adult 0.4084733 0.4150 0.05933896         0.36000         0.45000
## 10    Male   old 0.4728774 0.4750 0.05403239         0.43500         0.51500
## 11  Female   old 0.4707658 0.4750 0.05995149         0.42500         0.52000
## 12 Uvenile   old 0.4310417 0.4375 0.05593045         0.38750         0.46875
\end{verbatim}

\begin{Shaded}
\begin{Highlighting}[]
\KeywordTok{print}\NormalTok{(Sum_Wheight)}
\end{Highlighting}
\end{Shaded}

\begin{verbatim}
##        Sex   Age      mean  median         sd first_quant.25% third_quant.75%
## 1     Male  baby 0.1150909 0.09400 0.07379391        0.084750        0.111250
## 2   Female  baby 0.1622500 0.16425 0.06848905        0.125000        0.201500
## 3  Uvenile  baby 0.1247284 0.10575 0.06922868        0.073000        0.151875
## 4     Male young 0.8643621 0.87600 0.39562426        0.563000        1.138500
## 5   Female young 0.9322975 0.93675 0.37165473        0.655750        1.198750
## 6  Uvenile young 0.4340480 0.40000 0.23134113        0.243250        0.584000
## 7     Male adult 1.0821342 1.08900 0.39576956        0.798750        1.367000
## 8   Female adult 1.1130830 1.11325 0.40040442        0.804625        1.394250
## 9  Uvenile adult 0.7362405 0.71100 0.30658161        0.504000        0.904750
## 10    Male   old 1.1621132 1.12525 0.33882065        0.925000        1.366000
## 11  Female   old 1.1515315 1.09150 0.38091347        0.859250        1.451000
## 12 Uvenile   old 0.9436875 0.93500 0.33954035        0.717625        1.203000
\end{verbatim}

Построим график зависимости переменной Whole\_weight от переменной
Diameter для моллюсков разного пола. Как мы видим из графика,
зависимость между переменными близка к линейной.

\begin{Shaded}
\begin{Highlighting}[]
\KeywordTok{ggplot}\NormalTok{(Mollusca, }\KeywordTok{aes}\NormalTok{(}\DataTypeTok{x =}\NormalTok{ Diameter, }\DataTypeTok{y =}\NormalTok{ Whole_weight, }\DataTypeTok{col =}\NormalTok{ Sex))}\OperatorTok{+}
\StringTok{  }\KeywordTok{geom_smooth}\NormalTok{()}\OperatorTok{+}
\StringTok{  }\KeywordTok{labs}\NormalTok{(}\DataTypeTok{title =} \StringTok{"Whole weight vs Diameter dependance in different sexes"}\NormalTok{)}\OperatorTok{+}
\StringTok{  }\KeywordTok{theme}\NormalTok{(}\DataTypeTok{legend.position =} \StringTok{"bottom"}\NormalTok{)}
\end{Highlighting}
\end{Shaded}

\includegraphics{Project_files/figure-latex/unnamed-chunk-16-1.pdf}

Расчитаем среднее значение и sd для переменных Length и Whole\_weight.
Затем определим, чему равен процент моллюсков, чья высота не превышает
0.165, а также 93 квантиль для переменной Length (Ответы на 3-5 вопрос
из задания)

\begin{Shaded}
\begin{Highlighting}[]
\NormalTok{Mollusca }\OperatorTok\StringTok{ }\KeywordTok{group_by}\NormalTok{(Sex) }\OperatorTok\StringTok{ }\KeywordTok{select}\NormalTok{(Length, Whole_weight) }\OperatorTok\StringTok{ }\KeywordTok{summarise_each}\NormalTok{(}\KeywordTok{funs}\NormalTok{(mean, sd)) }\CommentTok{# 3}
\end{Highlighting}
\end{Shaded}

\begin{verbatim}
## # A tibble: 4 x 5
##   Sex     Length_mean Whole_weight_mean Length_sd Whole_weight_sd
##   <fct>         <dbl>             <dbl>     <dbl>           <dbl>
## 1 Male          0.558             0.955    0.0954           0.413
## 2 Female        0.576             1.02     0.0838           0.398
## 3 Uvenile       0.438             0.446    0.0982           0.277
## 4 <NA>          0.580             0.974   NA               NA
\end{verbatim}

\begin{Shaded}
\begin{Highlighting}[]
\NormalTok{small <-}\StringTok{ }\NormalTok{Mollusca }\OperatorTok\StringTok{ }\KeywordTok{filter}\NormalTok{(Height }\OperatorTok{<}\StringTok{ }\FloatTok{0.165}\NormalTok{)}
\NormalTok{percent_of_small <-}\StringTok{ }\KeywordTok{nrow}\NormalTok{(small) }\OperatorTok{/}\StringTok{ }\KeywordTok{nrow}\NormalTok{(Mollusca) }\OperatorTok{*}\DecValTok{100} \CommentTok{#4}

\NormalTok{quantile_vec <-}\StringTok{ }\KeywordTok{quantile}\NormalTok{(Mollusca}\OperatorTok{$}\NormalTok{Length,}\KeywordTok{c}\NormalTok{(}\DecValTok{0}\NormalTok{,}\FloatTok{0.25}\NormalTok{,}\FloatTok{0.5}\NormalTok{,}\FloatTok{0.75}\NormalTok{, }\FloatTok{0.921}\NormalTok{, }\DecValTok{1}\NormalTok{))}
\KeywordTok{print}\NormalTok{(quantile_vec[}\DecValTok{5}\NormalTok{])}
\end{Highlighting}
\end{Shaded}

\begin{verbatim}
## 92.1% 
##  0.66
\end{verbatim}

Создадим стандартизированную переменную для длины и сравним диаметр двух
групп животных, молодые и взрослые моллюски. Расчитаем тест Мана-Уитни
для выяснения достоверности отличий по диаметру между моллюсками разного
возраста. Как мы видим из данных теста, диаметры этих двух групп
достоверно отличаются друг от друга (ответ на 6 и 7 вопрос задания)

\begin{Shaded}
\begin{Highlighting}[]
\NormalTok{Mollusca}\OperatorTok{$}\NormalTok{Lenght_z_scores <-}\StringTok{ }\KeywordTok{scale}\NormalTok{(Mollusca}\OperatorTok{$}\NormalTok{Length, }\DataTypeTok{center =} \OtherTok{TRUE}\NormalTok{, }\DataTypeTok{scale =} \OtherTok{TRUE}\NormalTok{) }\CommentTok{#6}
\NormalTok{ y_Diam <-}\StringTok{ }\NormalTok{Mollusca }\OperatorTok\StringTok{ }\KeywordTok{filter}\NormalTok{(Rings }\OperatorTok{==}\StringTok{ }\DecValTok{5}\NormalTok{) }\OperatorTok\StringTok{ }\KeywordTok{select}\NormalTok{(Diameter) }
\NormalTok{ o_Diam <-}\StringTok{ }\NormalTok{Mollusca }\OperatorTok\StringTok{ }\KeywordTok{filter}\NormalTok{(Rings }\OperatorTok{==}\StringTok{ }\DecValTok{15}\NormalTok{) }\OperatorTok\StringTok{ }\KeywordTok{select}\NormalTok{(Diameter) }
 \KeywordTok{wilcox.test}\NormalTok{(o_Diam}\OperatorTok{$}\NormalTok{Diameter, y_Diam}\OperatorTok{$}\NormalTok{Diameter) }\CommentTok{#7}
\end{Highlighting}
\end{Shaded}

\begin{verbatim}
## 
##  Wilcoxon rank sum test with continuity correction
## 
## data:  o_Diam$Diameter and y_Diam$Diameter
## W = 10199, p-value < 2.2e-16
## alternative hypothesis: true location shift is not equal to 0
\end{verbatim}

\hypertarget{ux43eux446ux435ux43dux43aux430-ux434ux43eux441ux442ux43eux432ux435ux440ux43dux43eux441ux442ux438-ux440ux430ux437ux43bux438ux447ux438ux439-ux432-ux43fux43eux43aux430ux437ux430ux442ux435ux43bux44fux445-ux432-ux437ux430ux432ux438ux441ux438ux43cux43eux441ux442ux438-ux43eux442-ux432ux43eux437ux440ux430ux441ux442ux430-ux438-ux43fux43eux43bux430}{%
\subsubsection{Оценка достоверности различий в показателях в зависимости
от возраста и
пола}\label{ux43eux446ux435ux43dux43aux430-ux434ux43eux441ux442ux43eux432ux435ux440ux43dux43eux441ux442ux438-ux440ux430ux437ux43bux438ux447ux438ux439-ux432-ux43fux43eux43aux430ux437ux430ux442ux435ux43bux44fux445-ux432-ux437ux430ux432ux438ux441ux438ux43cux43eux441ux442ux438-ux43eux442-ux432ux43eux437ux440ux430ux441ux442ux430-ux438-ux43fux43eux43bux430}}

Построим графики зависимости диаметра и веса от возраста и пола. С
помощью критерия Крускела-Уоллиса оценим разницу между полами по
показателям в каждой возрастной группе. В случае достоверных различий по
критерию Крускела-Уоллиса, оценим различия между группами с помощью
теста Данна.

\begin{Shaded}
\begin{Highlighting}[]
\NormalTok{age_groups <-}\StringTok{ }\KeywordTok{c}\NormalTok{(}\StringTok{"baby"}\NormalTok{, }\StringTok{"young"}\NormalTok{,}\StringTok{"adult"}\NormalTok{,}\StringTok{"old"}\NormalTok{)}
\NormalTok{list_kt_D <-}\StringTok{ }\KeywordTok{lapply}\NormalTok{(age_groups, }\ControlFlowTok{function}\NormalTok{(i) \{}\KeywordTok{kruskal.test}\NormalTok{(Mollusca}\OperatorTok{$}\NormalTok{Diameter }\OperatorTok{~}\StringTok{ }\NormalTok{Mollusca}\OperatorTok{$}\NormalTok{Sex, }\DataTypeTok{subset=}\NormalTok{ Mollusca}\OperatorTok{$}\NormalTok{Age }\OperatorTok{==}\NormalTok{i )\})}
\KeywordTok{names}\NormalTok{(list_kt_D) <-}\StringTok{ }\NormalTok{age_groups}
\NormalTok{lst_Dt_D <-}\StringTok{ }\KeywordTok{lapply}\NormalTok{(age_groups, }\ControlFlowTok{function}\NormalTok{(i) \{}\KeywordTok{DunnTest}\NormalTok{(Mollusca}\OperatorTok{$}\NormalTok{Diameter }\OperatorTok{~}\StringTok{ }\NormalTok{Mollusca}\OperatorTok{$}\NormalTok{Sex, }\DataTypeTok{subset=}\NormalTok{ Mollusca}\OperatorTok{$}\NormalTok{Age }\OperatorTok{==}\NormalTok{i )\})}
\KeywordTok{names}\NormalTok{(lst_Dt_D) <-}\StringTok{ }\NormalTok{age_groups}
\KeywordTok{print}\NormalTok{(lst_Dt_D)}
\end{Highlighting}
\end{Shaded}

\begin{verbatim}
## $baby
## 
##  Dunn's test of multiple comparisons using rank sums : holm  
## 
##                mean.rank.diff   pval    
## Female-Male          33.47727 0.3914    
## Uvenile-Male          5.31348 0.6569    
## Uvenile-Female      -28.16379 0.3914    
## ---
## Signif. codes:  0 '***' 0.001 '**' 0.01 '*' 0.05 '.' 0.1 ' ' 1
## 
## 
## $young
## 
##  Dunn's test of multiple comparisons using rank sums : holm  
## 
##                mean.rank.diff    pval    
## Female-Male          128.8243 0.00064 ***
## Uvenile-Male        -735.8740 < 2e-16 ***
## Uvenile-Female      -864.6983 < 2e-16 ***
## ---
## Signif. codes:  0 '***' 0.001 '**' 0.01 '*' 0.05 '.' 0.1 ' ' 1
## 
## 
## $adult
## 
##  Dunn's test of multiple comparisons using rank sums : holm  
## 
##                mean.rank.diff    pval    
## Female-Male          37.16929  0.0716 .  
## Uvenile-Male       -258.44650 1.2e-15 ***
## Uvenile-Female     -295.61579 < 2e-16 ***
## ---
## Signif. codes:  0 '***' 0.001 '**' 0.01 '*' 0.05 '.' 0.1 ' ' 1
## 
## 
## $old
## 
##  Dunn's test of multiple comparisons using rank sums : holm  
## 
##                mean.rank.diff   pval    
## Female-Male         -1.956655 0.8362    
## Uvenile-Male       -46.558569 0.0094 ** 
## Uvenile-Female     -44.601914 0.0094 ** 
## ---
## Signif. codes:  0 '***' 0.001 '**' 0.01 '*' 0.05 '.' 0.1 ' ' 1
\end{verbatim}

\begin{Shaded}
\begin{Highlighting}[]
\KeywordTok{ggplot}\NormalTok{(Mollusca, }\KeywordTok{aes}\NormalTok{(}\DataTypeTok{x =}\NormalTok{ Age, }\DataTypeTok{y =}\NormalTok{ Diameter, }\DataTypeTok{col =}\NormalTok{ Sex))}\OperatorTok{+}
\StringTok{    }\KeywordTok{geom_boxplot}\NormalTok{()}\OperatorTok{+}
\StringTok{    }\KeywordTok{labs}\NormalTok{(}\DataTypeTok{title =}\StringTok{"Diameter change across Sex and Age"}\NormalTok{ )}
\end{Highlighting}
\end{Shaded}

\includegraphics{Project_files/figure-latex/unnamed-chunk-19-1.pdf}

\begin{Shaded}
\begin{Highlighting}[]
\NormalTok{list_kt_W <-}\StringTok{ }\KeywordTok{lapply}\NormalTok{(age_groups, }\ControlFlowTok{function}\NormalTok{(i) \{ }\KeywordTok{kruskal.test}\NormalTok{(Mollusca}\OperatorTok{$}\NormalTok{Whole_weight }\OperatorTok{~}\StringTok{ }\NormalTok{Mollusca}\OperatorTok{$}\NormalTok{Sex, }\DataTypeTok{subset=}\NormalTok{ Mollusca}\OperatorTok{$}\NormalTok{Age }\OperatorTok{==}\NormalTok{i )\})}
\KeywordTok{names}\NormalTok{(list_kt_W) <-}\StringTok{ }\NormalTok{age_groups}
\NormalTok{lst_Dt_W <-}\StringTok{ }\KeywordTok{lapply}\NormalTok{(age_groups, }\ControlFlowTok{function}\NormalTok{(i) \{ }\KeywordTok{DunnTest}\NormalTok{(Mollusca}\OperatorTok{$}\NormalTok{Whole_weight }\OperatorTok{~}\StringTok{ }\NormalTok{Mollusca}\OperatorTok{$}\NormalTok{Sex, }\DataTypeTok{subset=}\NormalTok{Mollusca}\OperatorTok{$}\NormalTok{Age }\OperatorTok{==}\NormalTok{i)\})}
\KeywordTok{names}\NormalTok{(lst_Dt_W) <-}\StringTok{ }\NormalTok{age_groups}
\KeywordTok{print}\NormalTok{(lst_Dt_W)}
\end{Highlighting}
\end{Shaded}

\begin{verbatim}
## $baby
## 
##  Dunn's test of multiple comparisons using rank sums : holm  
## 
##                mean.rank.diff   pval    
## Female-Male         32.363636 0.4327    
## Uvenile-Male         7.764498 0.5167    
## Uvenile-Female     -24.599138 0.4327    
## ---
## Signif. codes:  0 '***' 0.001 '**' 0.01 '*' 0.05 '.' 0.1 ' ' 1
## 
## 
## $young
## 
##  Dunn's test of multiple comparisons using rank sums : holm  
## 
##                mean.rank.diff   pval    
## Female-Male          123.4439 0.0011 ** 
## Uvenile-Male        -791.3677 <2e-16 ***
## Uvenile-Female      -914.8116 <2e-16 ***
## ---
## Signif. codes:  0 '***' 0.001 '**' 0.01 '*' 0.05 '.' 0.1 ' ' 1
## 
## 
## $adult
## 
##  Dunn's test of multiple comparisons using rank sums : holm  
## 
##                mean.rank.diff   pval    
## Female-Male          21.36608 0.3005    
## Uvenile-Male       -284.68569 <2e-16 ***
## Uvenile-Female     -306.05177 <2e-16 ***
## ---
## Signif. codes:  0 '***' 0.001 '**' 0.01 '*' 0.05 '.' 0.1 ' ' 1
## 
## 
## $old
## 
##  Dunn's test of multiple comparisons using rank sums : holm  
## 
##                mean.rank.diff   pval    
## Female-Male         -2.866055 0.7621    
## Uvenile-Male       -39.226415 0.0384 *  
## Uvenile-Female     -36.360360 0.0410 *  
## ---
## Signif. codes:  0 '***' 0.001 '**' 0.01 '*' 0.05 '.' 0.1 ' ' 1
\end{verbatim}

\begin{Shaded}
\begin{Highlighting}[]
\KeywordTok{ggplot}\NormalTok{(Mollusca, }\KeywordTok{aes}\NormalTok{(}\DataTypeTok{x =}\NormalTok{ Age, }\DataTypeTok{y =}\NormalTok{ Whole_weight, }\DataTypeTok{col =}\NormalTok{ Sex))}\OperatorTok{+}
\StringTok{   }\KeywordTok{geom_boxplot}\NormalTok{()}\OperatorTok{+}
\StringTok{   }\KeywordTok{labs}\NormalTok{(}\DataTypeTok{title =}\StringTok{"Whole weight change across Sex and Age"}\NormalTok{ )}
\end{Highlighting}
\end{Shaded}

\includegraphics{Project_files/figure-latex/unnamed-chunk-20-1.pdf} Как
можно понять из приведенных даных, диаметр и вес моллюсков значимо
отличаются во всех возрастных группах, кроме совсем молодых животных
(группа ``baby'') между половозрелыми и неполовозрелыми животными.
Отличия между самцами и самками наблюдаются только в группе юных
животных (``young'').

\hypertarget{ux432ux44bux432ux43eux434ux44b}{%
\subsubsection{Выводы}\label{ux432ux44bux432ux43eux434ux44b}}

Подводя итог всему вышеперичисленному можно сделать несколько выводов:\\

Numbered list:

\begin{enumerate}
\def\labelenumi{\arabic{enumi}.}
\tightlist
\item
  Переменные, описывающие вес (Whole\_weight, Shucked\_weight,
  Viscera\_weight, Shell\_weight), также как ипеременные, описывающие
  размер (Diameter, Length, Hight) демонстрируют сильную корреляцию
  между собой.\\
\item
  Переменные, характеризующие размер и вес также сильно коррелируют друг
  с другом.\\
\item
  Возраст демоснтрирует слабую и среднюю корреляцию с переменными
  размера и веса. Наибольшая корреляция наблюдается между возрастом и
  весом раковины моллюска. 4. При оценке влияния пола в различных
  возрастных группах было выявлено, что достоверная разница в диаметре и
  общем весе моллюска наблюдается между половозрелыми и неполовозрелыми
  особями во всех возрастных группах, кроме совсем молодых особей
  (моложе 6 лет). Показатели неполовозрелых особей значительно ниже, чем
  показатели половозрелых.
\item
  Достоверная разница в диаметре и общем весе между самцами и самками
  показана только в группе молодых животных (6-10 лет). Показатели самок
  выше, чем показатели самцов.
\item
  На основе имеющихся данных нельзя построить линейную модель
  зависимости веса моллюсков от их размера, возраста или пола т.к.
  абсолютно все переменные нашего датафрема характеризуются ненормальным
  распределением, более того, дисперсия признаков увеличивается в
  зависимости от возраста и пола, что также не позволяет нам построить
  адекватную линейную модель.
\end{enumerate}

\end{document}
